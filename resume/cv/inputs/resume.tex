% Copyright 2013 Christophe-Marie Duquesne <chmd@chmd.fr>
% Copyright 2014 Mark Szepieniec <http://github.com/mszep>
% 
% ConText style for making a resume with pandoc. Inspired by moderncv.
% 
% This CSS document is delivered to you under the CC BY-SA 3.0 License.
% https://creativecommons.org/licenses/by-sa/3.0/deed.en_US

\startmode[*mkii]
  \enableregime[utf-8]  
  \setupcolors[state=start]
\stopmode

\setupcolor[hex]
\definecolor[titlegrey][h=757575]
\definecolor[sectioncolor][h=110061]
\definecolor[rulecolor][h=110061]

% Enable hyperlinks
\setupinteraction[state=start, color=sectioncolor]

\setuppapersize [A4][A4]
\setuplayout    [width=middle, height=middle,
                 backspace=20mm, cutspace=0mm,
                 topspace=10mm, bottomspace=20mm,
                 header=0mm, footer=0mm]

%\setuppagenumbering[location={footer,center}]

\setupbodyfont[11pt, helvetica]

\setupwhitespace[medium]

\setupblackrules[width=31mm, color=rulecolor]

\setuphead[chapter]      [style=\tfd, page=no]
\setuphead[section]      [style=\tfd\bf, color=titlegrey, align=middle, page=no]
\setuphead[subsection]   [style=\tfb\bf, color=sectioncolor, align=right,
                          before={\leavevmode\blackrule\hspace}, page=no]
\setuphead[subsubsection][style=\bf, page=no]

\setuphead[chapter, section, subsection, subsubsection][number=no]

\definemeasure[cslhangindent][1.5em]
\definestartstop
  [cslreferences]
  [before={%
     \setupnarrower[left=\measure{cslhangindent}]
     \startnarrower[left]%
     \setupindenting[-\leftskip,yes,first]%
     \doindentation%
   },
   after=\stopnarrower]

%\setupdescriptions[width=10mm]

\definedescription
  [description]
  [headstyle=bold, style=normal,
   location=hanging, width=18mm, distance=14mm, margin=0cm]

\setupitemize[autointro, packed]    % prevent orphan list intro
\setupitemize[indentnext=no]

\defineitemgroup[enumerate]
\setupenumerate[each][fit][itemalign=left,distance=.5em,style={\feature[+][default:tnum]}]

\setupfloat[figure][default={here,nonumber}]
\setupfloat[table][default={here,nonumber}]

\setuptables[textwidth=max, HL=none]
\setupxtable[frame=off,option={stretch,width}]

\setupthinrules[width=15em] % width of horizontal rules

\setupdelimitedtext
  [blockquote]
  [before={\setupalign[middle]},
   indentnext=no,
  ]


\starttext
\startalignment[right]
  \blank[2*big]
  {\tfd Mathias Thibaut Louboutin}
  \blank[3*medium]
  {\tfa \crlf
733 York View Dr, Auburn, 30011, GA\crlf
mathias.louboutin@gmail.com\crlf
404-451-6131\crlf
\useURL[url1][mloubout.github.io]\from[url1]\crlf
\goto{Github}[url(https://github.com/mloubout)]\crlf
\goto{@louboutjunior}[url(https://twitter.com/Louboutjunior)]}
  \blank[3*medium]
\stopalignment

\startsectionlevel[title={Experience},reference={experience}]

\startdescription{March 2023-*}
  {\bf Computational Imaging Scientist}: Oak Ridge National Laboratory,
  Oak Ridge, TN {\em Scientific engagmenent, INCITE }

  {\em Medical Imaging}

  {\em HPC (Frontier, Summit, \ldots{})}
\stopdescription

\startdescription{July 2020-- February 2023}
  {\bf Postdoctoral Fellow}: Georgia Institute of Technology, Atlanta,
  GA

  {\em High performance/low memory randomized linear algebra for
  backpropagation based inverse problems}

  {\em Cloud HPC for separable problems (task parallel)}

  {\em Supervising the PhD and MSc students}

  {\em Managing and developping the software stack for the Lab
  (\goto{slimgroup}[url(https://github.com/slimgroup)])}

  {\em Machine learning for geophysical and medical wave-equation based
  inverse problems}

  {\em HPC for machine learning}

  {\em Geological Carbon Storage seismic monitoring}
\stopdescription

\stopsectionlevel

\startsectionlevel[title={Computational
experience},reference={computational-experience}]

\startdescription{Open Source}
  {\bf \goto{Devito}[url(https://github.com/devitocodes/devito)]}: A
  symbolic domain specific language (DSL) for stencil computation with
  just-in-time compilation and code generation. Achieves state of the
  art performance while providing a high-level mathematical interface to
  the users for the development of stencil based applications.

  {\bf \goto{JUDI}[url(https://github.com/slimgroup/JUDI.jl)]}: Linear
  algebra high level API for wave-equation based inversion. This pacage
  is built on top of Devito to have high performance wave-equation
  solvers. A new additional Azure batch extension was developped for
  scalability
  \goto{JUDI4Cloud}[url(https://github.com/slimgroup/JUDI.jl)].

  {\bf \goto{XConv}[url(https://github.com/slimgroup/XConv)]}: High
  performance low memory convolutional layer. This repository implements
  both in julia (for Flux.jl) and in python (for pytorch) a
  convolutional layer that has virtually a zero memory imprnt for
  training using randomized linear algebra to compute an unbiased
  estimate of the gradient with respect to the weights. Additionaly, a
  byte only implementation of the ReLU layer leads to memry reduction by
  a factor of X2 for full networks.

  {\bf \goto{dfno}[url(https://github.com/slimgroup/dfno)]}: Model
  parallel (MPI model decomposition) implementation of Fourier Neural
  Operators for PyTorch. Extension of distdl, a model parallel extension
  of PyTorch.

  {\bf \goto{InvertibleNetworks.jl}[url(https://github.com/slimgroup/InvertibleNetworks.jl)]}:
  Native Julia implementation of invertible networks for variational
  inference, generative models and normalizing flows.
\stopdescription

\startdescription{Programming Languages}
  {\bf Python:} Main programming language for the development of
  \goto{Devito}[url(https://github.com/devitocodes/devito)] and machine
  learning applications.

  {\bf Julia:} Heavy development of research software at Georgia Tech
  (\goto{slimgroup}[url(https://github.com/slimgroup)]) in Julia

  {\bf docker} Developped and automatized the deployement of
  \goto{Devito}[url(https://github.com/devitocodes/devito)] and
  \goto{JUDI}[url(https://github.com/slimgroup/JUDI.jl)] images through
  CI (github actions).

  Knowledge of {\bf C}, {\bf Linux}, {\bf Bash}, {\bf PyTorch},
  {\bf Azure}, {\bf Latex}, {\bf Markdown}, {\bf Matlab}, {\bf MPI},
  {\bf OPenMP}, {\bf OpenACC}
\stopdescription

\startdescription{HPC}
  {\bf \goto{Devito}[url(https://github.com/devitocodes/devito)]:} Weak
  and strong scaling benchmarks of
  \goto{Devito}[url(https://github.com/devitocodes/devito)] on
  on-premise (Imperical college) and Cloud (Azure) hardware.

  {\bf \goto{JUDI}[url(https://github.com/slimgroup/JUDI.jl)]:}
  Implementation and deployment at scale of
  \goto{JUDI}[url(https://github.com/slimgroup/JUDI.jl)] on clusters and
  Azure Batch (up 300 nodes).

  {\bf Optimum (2015-2018):} Early PhD 50 nodes cluster. Development of
  parallel Matlab seismic inverse problem algorithms (FWI/RTM).

  {\bf YEMOJA (2017-2018):} Part of a collaboration with SENAI-CIMANTEC.
  Scaling of our Matlab and Julia framework to hundred of nodes.

  {\bf Cloud (2018-):} Serverless and clusterless framework for task
  parallel inverse problems on AWS and Azure.

  {\bf Perlmutter (2022-):} Scaling of MPI-parallel Fourier Neural
  Operator on Perlmutter (and previously Summit).
\stopdescription

\stopsectionlevel

\startsectionlevel[title={Education},reference={education}]

\startdescription{2018--2020}
  {\bf PhD, Computer Science}; Georgia Institute of Technology, Atlanta,
  GA

  {\em Thesis title: Modeling for inversion in exploration geophysics}
  \goto{Link}[url(https://slim.gatech.edu/content/modeling-inversion-exploration-geophysics)]

  {\em Numerical and computational methods for large scale simulation
  based inverse problems and machine learning}
\stopdescription

\startdescription{2013--2018}
  {\bf PhD, Earth Science}; University of British Columbia, Canada

  {\em Transfered to Georgia Institute of Technology in January 2018
  following my supervior new position there.}
\stopdescription

\startdescription{2016 Feb-Aug}
  {\bf Visiting PhD, Computer Science}; Impertial College London, UK

  {\em Automatic code generation for geophysical exploration
  applications with finite differences}
\stopdescription

\startdescription{2011--2013}
  {\bf MSc, Applied Mathematics}; Universite de Rennes 1, Rennes, France

  {\em Valedictorian}

  {\em Required coursework}: Calculus, Numerical Methods, PDE
  Resolution, Opti- mization, C/C++ Computing, Mathematics Modeling and
  Simulation, Finite Element Method*

  {\em Elective coursework}: Fluid Mechanics, Continuum Mechanics and
  Thermo- mechanics, Bio-mechanics, Geophysics Modeling*
\stopdescription

\startdescription{2008--2011}
  {\bf BSc, Aeronautical Engineering,} ENSICA-ISAE, Toulouse, France

  {\em Leading French Aeronautical Engineering School.}

  {\em Required coursework}: Mathematics, Mechanics, Continuum
  Mechanics, Struc- tures Mechanics, Signal Processing, Thermodynamics,
  Fluid Mechanics, Java progamming*

  {\em Elective coursework}: Estimation Methods, Earth Observation
  Satellites, Microwaves Processing*
\stopdescription

\startdescription{2006--2008}
  {\bf Classe Preparatoires}; Lycee Chateaubriand, Rennes, France

  {\em Advanced undergraduate preparatory program for national ranking
  entry exam.}
\stopdescription

\stopsectionlevel

\startsectionlevel[title={Internships},reference={internships}]

\startdescription{Summer 2013}
  {\bf Research internship}; ONERA, Toulouse, France

  {\em Scattering patterns of atmospheric dust clouds analysis with the
  Discrete Dipole Approximation (DDA) method.}
\stopdescription

\startdescription{Summer 2012}
  {\bf Research internship}; INRIA, Grenoble, France

  {\em Intern in NANO-D department at INRIA-Grenoble. L2-SVM for protein
  interactions. Runtime and accuracy improvement of the C implementation
  and algorithmic development.}
\stopdescription

\startdescription{Summer 2011}
  {\bf Internship, Aeroconseil}, Toulouse, France

  {\em Developed an interface for aerodynamics calculus in JAVA. Reading
  and implementation of Excel and Scilab scripts through the interface.}
\stopdescription

\stopsectionlevel

\startsectionlevel[title={Additonal
skills},reference={additonal-skills}]

\startitemize
\item
  Languages:

  \startitemize[packed]
  \item
    French (native speaker)
  \item
    English (Advanced, PhD in USA)
  \stopitemize
\item
  Miscelanous CS:

  \startitemize[packed]
  \item
    Linux, Shell script, Latex, Markdown, Github, Unix, Matlab
  \stopitemize
\stopitemize

\stopsectionlevel

\stoptext
